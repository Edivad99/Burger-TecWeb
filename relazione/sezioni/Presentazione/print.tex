Per la stampa abbiamo eliminato tutte le funzioni interattive lasciando solo il contenuto di interesse.
Questo perché nessuna delle form richiedeva troppi dati(non serve stampare la form per preparare tutti i dati da inserire in seguito).\\
Sono stati eliminati background, colori, sottolineature non necessarie, immagini non importanti, eventuali aiuti nascosti, le icone del menu a tendina, il breadcrumb ed è stato utilizzato uno stile \emph{serif} per la scrittura.\\
È stata presa la decisione di lasciare alcuni link sottolineati per far sapere all'utente la presenza di eventuali pagine che non ha visitato ma che potrebbero interessarli. %Da rivedere parte link
Sono state modificate le dimensioni di alcune scritte e lo spazio in alcune pagine è stato riadattato.\\
È stata fatta la scelta di mantenere le immagini dei panini nella pagina \emph{Menu}; 
questo perché la pagina viene usata come vetrina per mostrare i prodotti che vendiamo.
Inoltre al cliente potrebbe interessare stampare la vetrina dei prodotti per poter decidere in seguito.\\
La foto dei panini è stata tolta dalla pagina \emph{Panino} poiché si è ritenuto che se un utente stampa questa pagina ha già deciso che panino prendere e può essere interessato a stampare per diversi motivi, ad esempio:
\begin{itemize}
	\item Potrebbe stampare la pagina per potersi ricordare il nome del panino;
	\item Potrebbe essere interessato alla lista ingredienti in caso dovesse ricordarsi di richiedere l'eliminazione di un ingrediente per il suo ordine;
	\item Potrebbe voler stampare un commento ritenuto offensivo e avere una prova cartacea;
\end{itemize}