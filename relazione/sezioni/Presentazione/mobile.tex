\subsection{Mobile}
Per facilitarne l'utilizzo e rendere accessibile il sito da \emph{mobile} è stato deciso di apportare alcune modifiche rispetto alla versione \emph{Dekstop}.\\
È stato necessario utilizzare due punti di rottura per le dimensioni dello schermo:
\begin{itemize}
	\item Un max-width di 730px
	\item Un max-width di 694px
\end{itemize}
In base al punto di rottura avvengono determinate modifiche allo stile delle pagine:
\begin{itemize}
	\item \textbf{730px}
	\begin{itemize}
		\item L'unica modifica, a questo punto di rottura, riguarda la pagina \emph{Menu} dove la sezione verticale a sinistra del \emph{contenutoGenerale}, usata per la selezione della categoria, viene spostata al di sopra del \emph{contenutoGenerale};
Viene trasformata in una sezione orizzontale tenendo la linea di separazione tra questa sezione e il \emph{contenutoGenerale}, presente anche nella versione \emph{Desktop}. 
In questo modo viene facilitata la visione e l'utilizzo da \emph{mobile}.
	\end{itemize}
	\item \textbf{694px}\\
	A questo punto di rottura avvengono diverse modifiche:
	\begin{itemize}
		\item Il logo del locale viene rimosso;
		\item Il menu orizzontale viene trasformato in un menu a tendina(tramite \emph{CSS}), con apertura verticale, adattandosi all'utilizzo da \emph{mobile};
		\item La pagina \emph{Panino} viene riadattata per un utilizzo migliore da \emph{mobile};
		\item La pagina \emph{Contatti} viene riadattata per un utilizzo migliore da \emph{mobile}, verticalizzando il contenuto;
		\item La pagina \emph{Area Riservata} viene riadattata per un utilizzo migliore da \emph{mobile}, verticalizzando i vari bottoni per la gestione;
		\item La pagina \emph{gestioneEventi} viene riadattata per un utilizzo migliore da \emph{mobile}, verticalizzando le form.
	\end{itemize}
\end{itemize}