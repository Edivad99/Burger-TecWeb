È stato deciso di utilizzare il linguaggio \emph{HTML5} per diversi motivi:
\begin{itemize} %magari trovare altri motivi
	\item \emph{HTML5} è oramai supportato dalla maggior parte dei Browser Web;
	\item Dopo un sondaggio proposto alla clientela è stato riscontrato che il 70\% dei clienti sceglie fuori casa di mangiare da noi e quindi userebbero il telefono per visitare il sito;
	\item La maggior parte degli accessi al web avviene tramite mobile dove i browser sono spesso aggiornati.
\end{itemize}
Si tiene a precisare che in ogni caso viene rispettata la sintassi \emph{XHTML} e in caso il browser non supporti \emph{HTML5} il sito degraderà in modo elegante.\\
Inoltre si è scelto di utilizzare \emph{HTML5} per la sua fluidezza e funzionalità aggiuntive. %verificare
Esiste una pagina \emph{HTML5} per ogni pagina del sito.\\ %decidere se mettere o meno
Come linguaggio di stile è stato usato \emph{CSS3}.\\
È stata mantenuta la separazione tra struttura e presentazione utilizzando i seguenti accorgimenti: 
\begin{itemize}
    \item non sono stati usati tag di stile nelle pagine \emph{HTML};
    \item per la presentazione sono stati utilizzati file esterni senza utilizzare codice a cascata inline o embedded.
\end{itemize}
Ogni singola regola è stata valutata prima di utilizzarla, in base alla compatibilità dei browser.\\
In alcune parti del sito sono state utilizzate le flex invece delle grid poichè supportato maggiormente.\\
I fogli di stile utilizzati sono tre:
\begin{itemize}
	\item Un foglio style.css per le impostazioni di stile generale;
	\item Un foglio mobile.css per le impostazioni di stile che riguardano il mobile, separando i diversi punti di rottura; %magari impostare meglio frase
	\item Un foglio print.css per le impostazioni che riguardano la stampa.
\end{itemize}