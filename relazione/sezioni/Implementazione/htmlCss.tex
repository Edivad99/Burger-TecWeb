\subsection{HTML5 e CSS}
È stato deciso di utilizzare il linguaggio \emph{HTML5} per diversi motivi:
\begin{itemize} %magari trovare altri motivi
	\item \emph{HTML5} è oramai supportato dalla maggior parte dei browser web;
	\item Dopo un sondaggio proposto alla clientela, è stato riscontrato che il 70\% dei nostri clienti sceglie di mangiare da noi quando si trova fuori casa. 
	Quindi la maggior parte dei nostri clienti utilizzerebbe il telefono per visitare il sito;
	\item La maggior parte degli accessi al web avviene tramite mobile.
\end{itemize}
Gli ultimi due punti si concentrano sull'utilizzo di un dispositivo \emph{Mobile} come strumento per visitare il futuro sito; questo è importante perché la maggior parte dei dispositivi \emph{Mobile} utilizza browser aggiornati e quindi che supportano ampiamente \emph{HTML5}.\\
Si tiene a precisare che in ogni caso viene rispettata la sintassi \emph{XHTML} e in caso il browser non supporti \emph{HTML5} il sito degraderà in modo elegante.\\
Inoltre si è scelto di utilizzare \emph{HTML5} per la sua fluidità e funzionalità aggiuntive. %verificare
Esiste una pagina \emph{HTML5} per ogni pagina del sito.\\ %decidere se mettere o meno
Come linguaggio di stile è stato usato \emph{CSS3}.\\
È stata mantenuta la separazione tra struttura e presentazione utilizzando i seguenti accorgimenti: 
\begin{itemize}
    \item Non sono stati usati tag di stile nelle pagine \emph{HTML};
    \item Per la presentazione sono stati utilizzati file esterni senza utilizzare codice a cascata inline o embedded.
\end{itemize}
Ogni singola regola è stata valutata prima di utilizzarla, in base alla compatibilità dei browser.
In alcune parti del sito sono state utilizzate le flex invece delle grid poiché supportato maggiormente.\\
Nella maggior parte dei casi viene utilizzata come unità misura gli em o la percentuale.\\
Questo migliora l'accessibilità e garantisce una corretta visualizzazione delle pagine su tutti i formati di schermi.\\
I fogli di stile utilizzati sono tre:
\begin{itemize}
	\item Un foglio style.css per le impostazioni di stile generale;
	\item Un foglio mobile.css per le impostazioni di stile che riguardano il mobile, separando i diversi punti di rottura; %magari impostare meglio frase
	\item Un foglio print.css per le impostazioni che riguardano la stampa.
\end{itemize}