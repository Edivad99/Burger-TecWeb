È stato deciso di elaborare tutte le pagine in \emph{PHP}. Questa operazione è necessaria in quanto ogni 
pagina viene riportato il tasto di login se l'utente non è loggato, altrimenti ne viene mostrato l'username.\\
Il linguaggio \emph{PHP} viene utilizzato per leggere, ricavare, scrivere ed eliminare informazioni dal database.\\
Il sito è strutturato in questo modo: 
\begin{itemize}
    \item File \emph{HTML} per ogni singola pagina contenente tutte le parti statiche;
    \item I relativi file in \emph{PHP}, per elaborare le parti dinamiche e ritornare la pagina completa;
    \item File \emph{PHP} per la gestione di diverse funzionalità: inserimento o eliminazione di eventi, login e logout, la possibilità di votare o commentare un panino e la possibilità di cambiare password;
    \item File \emph{PHP} per l'interazione con il \emph{DB}.
\end{itemize}
In questo modo si mantiene la separazione tra struttura e comportamento, in quanto la struttura viene descritta nei file HTML, mentre il comportamento è gestito dai file \emph{PHP} e \emph{JS} come vedremo in seguito.\\

Nei file \emph{PHP} vengono eseguiti controlli lato client sull'input delle form controllando che i dati inseriti siano corretti e cercando di ridurre al minimo possibili tentativi di \emph{SQL injection}.\\
I metodi utilizzati per il controllo dell'input sono: 
    \begin{itemize}
        \item \emph{trim()}: elimina gli spazi prima e dopo la stringa in input;
        \item \emph{htmlentities()}: converte tutti i possibili caratteri speciali in entità \emph{HTML};
        \item \emph{strip\_tags()}: elimina tutti i possibili tag \emph{HTML};
        \item Inoltre viene utilizzata la funzione \emph{mysqli real escape string};
        \item Nella query per il login viene utilizzato \emph{BINARY} per rendere la form case sensitive;
    \end{itemize}