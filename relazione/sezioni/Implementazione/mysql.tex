Per il mantenimento e salvataggio dei dati e delle informazioni, si è deciso di utilizzare \emph{MySQL}\\
Per le funzioni che necessitano l'uso di \emph{MySQL} viene implementato l'estensione \emph{mysqli}\\
Il database utilizzato è descritto nel capitolo 3: \emph{Progettazione -> Database}.
\\
Da notare che alcuni campi per l'inserimento hanno una grandezza maggiore di quella limite imposta all'utente. Questo per assicurarci che anche usando tag HTML, dato che in \emph{MySQL} i simboli "<" e ">" vengono tradotti occupando più spazio, ci sia il rischio che l'input non venisse accettato dal DB portando ad un rifiuto dell'inserimento.