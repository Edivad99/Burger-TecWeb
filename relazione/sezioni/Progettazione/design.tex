Come già dichiarato sono state seguite due strategie per la creazione del sito: \emph{Mobile First} e l'attenzione fin dall'inizio all'accessibilità.\\
È stato scelto un \emph{layout a quattro pannelli} per la maggior parte delle pagine, che si adatta facilmente alla strategia di \emph{Mobile First}.\\
L'unica eccezione è la pagina \emph{Menu} che usa un \emph{layout a cinque pannelli} per poter inserire il menu categoria lateralmente.\\
I pannelli presenti nel layout sono, in ordine di presentazione:
\begin{enumerate}
	\item \textbf{header:} Contiene il logo e il nome del Locale e contiene il menu; 
	\item \textbf{breadcrumb:} Contiene il percorso della pagina corrente, serve ad orientare l'utente all'interno del sito;
	\item \textbf{contenutoGenerale:} Contiene le informazioni principali della pagina visitata;
	\item \textbf{footer:} Contiene le informazioni di contatto e la partita IVA del locale;
	\item \textbf{leftSideBar:} Nella versione \emph{Desktop} è utilizzata solo dalla pagina \emph{Menu}, si trova, come dice il nome, a sinistra del \emph{contenutoGenerale} e contiene la selezione della categoria dei panini. 
	Nella versione \emph{Mobile} si trova invece tra il pannello \emph{breadcrumb} e il pannello \emph{contenutoGenerale}.
\end{enumerate}