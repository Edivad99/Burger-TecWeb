\subsection{Obiettivi}
Il progetto \emph{Burgheria Padovana} si prefigge di rispettare le specifiche tecniche richieste:
\begin{itemize}
	\item \textbf{Rispettare gli standard:}\\ 
	È stato scelto di utilizzare \emph{HTML5} per il contenuto e \emph{CSS3} per la presentazione.
	\item \textbf{Separazione tra contenuto, presentazione e comportamento:}\\ 
	Per il contenuto è stato utilizzato \emph{HTML}, per la presentazione è stato usato \emph{CSS} e per il comportamento è stato utilizzato \emph{PHP} e \emph{JavaScript}.
	\item \textbf{Accessibilità:}\\ 
	Il sito è stato progettato per permettere a tutte le categorie di utenti di accedervi senza problemi.
	Per fare ciò si è scelto uno sviluppo del sito rispettando fin dal principio norme e pratiche per facilitare l'accessibilità.
	In questo modo si riduce il lavoro di controllo e validazione per quanto riguarda l'accessibilità evitando di dover riscrivere parti di codice che potrebbero mettere in difficoltà o creare disorientamento negli utenti.
	\item \textbf{Usabilità:}\\
	Il sito è stato progettato per permettere una navigazione fluida, semplice e tradizionale per evitare all'utente disorientamento e sovraccarico cognitivo.
	\item \textbf{PHP e DB:}\\
	Sono state utilizzate pagine in \emph{PHP} per poter interagire con il \emph{DB} ottenendo, scrivendo o eliminando dati. Sarà utilizzato un \emph{DB} dove verranno salvati eventuali dati inseriti dagli utenti.
	\item \textbf{Form:}\\
	Sono stati implementati controlli dell'input da parte dell'utente sia lato client (\emph{JavaScript}) sia lato server (\emph{PHP}).
	\item \textbf{Dispositivi:}\\
	Il sito è stato progettato per essere utilizzato da diversi dispositivi e a diverse dimensioni senza problemi. Per fare ciò si è utilizzata la pratica di sviluppo "\emph{Mobile First}" con la quale il progetto viene ideato pensando prima a dispositivi mobile.
\end{itemize}
Per rendere il sito accessibile e usabile sono state applicate le seguenti scelte: 
	\begin{itemize}
		\item Utilizzo dell'attributo \emph{alt} nelle immagini, per persone con disabilità visive.
		 Tale attributo permette di inserire delle brevi descrizioni alle immagini che verranno poi lette dai vari screen reader. 
		 La descrizione deve essere riportata solo se l'immagine è di contenuto e riporta informazioni importanti altrimenti, se l'immagine è solo di presentazione, la descrizione viene lasciata vuota.
		 Inoltre se si decide di implementare la descrizione, quest'ultima deve essere chiara ed esaustiva per non confondere l'utente.
		\item Utilizzo di \emph{aiuti nascosti}, per persone che utilizzano screen reader o navigazione tramite tastiera. 
		Gli \emph{aiuti nascosti} sono link invisibili all'utente ma visibile agli screen reader.
		 Sono utilizzati in diversi punti della pagina e permettono di saltare contenuti non interessanti e dirigersi direttamente a ciò che si cerca.
		\item Attenzione ai contrasti dei colori, per evitare problemi o disorientamento a utenti daltonici. 
		\item Per aiutare l'utente a capire dove si trova viene sottolineata nel menu la pagina in cui si trova l'utente; inoltre nella pagina \emph{Menu} viene sottolineata la categoria che l'utente sta visualizzando.
		\item Nelle form, quando l'utente sbaglia, vengono riportati avvisi di errore spiegando cosa è stato sbagliato facilitando la comprensione dell'utente ed evitando disorientamento.
		\item Non è stato necessaria l'implementazione di \emph{tabindex}, in quanto l'ordine seguito è logico e rispetta l'ordine dei contenuti della pagina.
		\item Nelle pagine \emph{Panino} e \emph{gestioneCommenti} è stato utilizzato il bottone \emph{torna su}; Questo bottone posto in fondo alla pagina, sotto ai commenti, è un link che permette all'utente di ritornare a inizio pagina evitando scroll inutili e velocizzando la navigazione. 
		\item Per la versione \emph{Mobile} è stata rispettata la "Comfort Zone" ossia il menu, le form, link o altre opzioni cliccabili sono state posizionate in zone comode per le dita, così che l'utente possa usufruirne con facilità. 
	\end{itemize}