\section{Introduzione}
Il progetto \emph{Burgheria Padovana} vuole implementare un sito Internet che offra la possibilità di fornire informazioni riguardo il suo punto vendita.\\
Il locale è aperto già da diversi anni e ha finalmente deciso di rinnovarsi creando un proprio sito internet.\\
Il sito dovrà contenere informazioni riguardanti i panini che offre, suddivisi nelle varie categorie: Pollo, Manzo e Speciali. 
Inoltre conterrà gli eventi a cui sarà possibile partecipare, la storia del locale, gli orari di apertura e come contattare i proprietari della Burgheria.\\
Il sito permette ad un utente privilegiato (admin) di inserire o eliminare i panini venduti, di inserire o eliminare gli eventi e di controllare ed eliminare i commenti degli altri utenti. Gli utenti normali (user) saranno visitatori 
con privilegi minimi per poter inserire od eliminare i propri commenti.\\
Entrambe le tipologie potranno visitare normalmente il sito e cambiare la password.\\
Gli \emph{user} dovranno effettuare la registrazione o il login prima di poter usufruire dei privilegi.\\
Inoltre deve essere garantita l'accessibilità in modo che chiunque possa navigare nel sito senza problemi e tranquillamente.\\
Terminato con l'accessibilità, si pone l'attenzione sull'usabilità: rispettando la separazione tra struttura, presentazione e comportamento e rispettando gli standard \emph{W3C} per quanto riguarda \emph{HTML} e \emph{CSS}.\\
Il sito dovrà garantire una navigazione fluida agli utenti evitando disorientamento e, nel caso accadesse, fornire supporto per indirizzarli al contenuto che cercavano.\\