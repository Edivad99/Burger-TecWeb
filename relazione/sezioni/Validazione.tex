La validazione è uno passo fondamentale del progetto, poiché serve a verificare che siano rispettati gli standard \emph{W3C} per quanto riguarda \emph{HTML} e \emph{CSS}. Infatti rispettare gli standard assicura un codice pulito, corretto e che favorisca l'accessibilità e l'usabilità.\\
Un sito che rispetta gli standard assicura:
\begin{itemize}
	\item un sito più accessibile e usabile;
	\item un alto livello di compatibilità tra i diversi browser, rendendo minime o nulle le differenze di visualizzazione del sito, sempre considerando la possibile versione del browser utilizzato dall'utenza a cui ci si riferisce;
	\item essendo valido non dovrebbe contenere errori e quindi la navigazione risulta più veloce e fluida;
	\item un sito valido e senza errori non viene interpretato dal browser "a modo suo" ma rispetta "il volere" degli sviluppatori comportandosi come previsto da questi ultimi; %rivedere
	\item un sito valido e senza errori viene indicizzato in modo positivo dal browser.  %rivedere
\end{itemize}
Per validare il sito sono stati utilizzati i seguenti strumenti:
\begin{itemize}
	\item \textbf{W3C HTML Validator}\\
	Indirizzo sito web: \emph{https://validator.w3.org/}\\
	È un servizio gratuito di \emph{W3C} che consente di validare le pagine \emph{HTML}. La validazione può avvenire in tre modi: attraverso il caricamento del file da validare, facendo copia e incolla del codice da validare o inserendo l'indirizzo della pagina se quest'ultima si trova online.\\
Se il codice non è valido, viene segnalato il numero di errori, il loro tipo e a quale riga e colonna della pagina sono stati trovati.
	\item \textbf{W3C CSS Validator}\\
	Indirizzo sito web: \emph{https://jigsaw.w3.org/css-validator/}\\
	È un servizio gratuito di \emph{W3C} che consente di validare il codice CSS.\\
	Anche in questo casi i metodi di validazione sono tre come nel sito precedente.\\
	La segnalazione degli errori viene gestita nell'identico modo del sito precedente.\\
	 \item In entrambi i siti, se le pagine e il codice risultano validi, riportano del codice HTML per poter utilizzare delle immagini, del \emph{W3C},che certificano la validazione del sito.
\end{itemize}