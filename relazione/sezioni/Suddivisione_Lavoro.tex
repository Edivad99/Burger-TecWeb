\section{Suddivisione del lavoro}
\begin{itemize}
	\item \textbf{Albiero Davide}
	\begin{itemize}
		\item Progettazione iniziale
		\item Sviluppo HTML
		\item Sviluppo CSS
		\item Sviluppo DB
		\item Sviluppo codice PHP
		\item Sviluppo codice JS
		\item Validazione delle pagine
		\item Test di accessibilità
		\item Test di usabilità
		\item Revisione Relazione
	\end{itemize}
	\item \textbf{Fincato Alessandro}
	\begin{itemize}
		\item Progettazione iniziale
		\item Sviluppo HTML
		\item Sviluppo CSS
		\item Sviluppo DB
		\item Sviluppo codice PHP
		\item Validazione delle pagine
		\item Test di accessibilità
		\item Test di usabilità
	\end{itemize}
	\item \textbf{Panighel Cristiano}
	\begin{itemize}
		\item Progettazione iniziale
		\item Sviluppo HTML
		\item Sviluppo CSS
		\item Validazione delle pagine
		\item Test di accessibilità
		\item Test di usabilità
	\end{itemize}
	\item \textbf{Tossuto Matteo}
	\begin{itemize}
		\item Progettazione iniziale
		\item Sviluppo HTML
		\item Sviluppo CSS
		\item Sviluppo DB
		\item Sviluppo codice PHP
		\item Sviluppo codice JS
		\item Validazione delle pagine
		\item Test di accessibilità
		\item Test di usabilità
		\item Stesura Relazione
	\end{itemize}
\end{itemize}

Queste sono le principali attività svolte dal singolo membro del gruppo.\\
Tutti i membri hanno partecipato sia allo sviluppo del sito mobile che desktop.\\
È stata usata una repository di \emph{GitHub}, per permettere uno svolgimento coordinato del lavoro attraverso l'utilizzo di \emph{Issue} e \emph{Pull Request}. Inoltre attraverso l'uso di \emph{GitHub} si può verificare il lavoro di ognuno dei membri.\\
La comunicazione è avvenuta tramite un gruppo \emph{Telegram}, tramite meeting \emph{Zoom} e lo scambio di commenti su \emph{GitHub}.\\
Attraverso questi strumenti veniva discusso come strutturare il progetto, eventuali scelte importanti, possibili soluzioni a bug e che strumenti utilizzare per la validazione e per i test.