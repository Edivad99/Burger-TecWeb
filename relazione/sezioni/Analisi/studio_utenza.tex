\subsection{Studio dell'utenza finale}
Il locale Burgheria Padovana offre un prodotto internazionale, comodo, veloce e buono.\\
Pertanto il sito è pensato per rivolgersi ad una categoria di utenti eterogenei, da consumatori abituali a chi vuole provare qualcosa di nuovo, da chi ha bisogno di mangiare qualcosa al volo a chi vuole gustarsi un buon pasto con calma.\\
Queste categorie di utenti, con privilegi minimi, verrà denominata come \emph{utente generico}.
Mentre l'utente con privilegi verrà denominato come \emph{amministratore}.\\ 
Entrambe le tipologie potranno accedere ai loro privilegi autenticandosi tramite form di login.\\
Essendo un utenza finale generica, sarà necessario utilizzare un linguaggio informale, semplice e comprensibile.
Allo stesso modo si andrà a creare un sito di struttura e layout semplici e simili ai modelli a cui l'\emph{utente generico} è abituato.
Si cercherà, quindi, di non rompere le convenzioni esterne e offrendo, indipendentemente dal browser o dal dispositivo utilizzato, una navigazione veloce e intuitiva.\\